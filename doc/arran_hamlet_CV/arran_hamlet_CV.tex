%!TEX TS-program = xelatex
%!TEX encoding = UTF-8 Unicode
% Awesome CV LaTeX Template for CV/Resume
%
% This template has been downloaded from:
% https://github.com/posquit0/Awesome-CV
%
% Author:
% Claud D. Park <posquit0.bj@gmail.com>
% http://www.posquit0.com
%
%
% Adapted to be an Rmarkdown template by Mitchell O'Hara-Wild
% 23 November 2018
%
% Template license:
% CC BY-SA 4.0 (https://creativecommons.org/licenses/by-sa/4.0/)
%
%-------------------------------------------------------------------------------
% CONFIGURATIONS
%-------------------------------------------------------------------------------
% A4 paper size by default, use 'letterpaper' for US letter
\documentclass[11pt, a4paper]{awesome-cv}

% Configure page margins with geometry
\geometry{left=1.4cm, top=.8cm, right=1.4cm, bottom=1.8cm, footskip=.5cm}

% Specify the location of the included fonts
\fontdir[fonts/]

% Color for highlights
% Awesome Colors: awesome-emerald, awesome-skyblue, awesome-red, awesome-pink, awesome-orange
%                 awesome-nephritis, awesome-concrete, awesome-darknight

\colorlet{awesome}{awesome-red}

% Colors for text
% Uncomment if you would like to specify your own color
% \definecolor{darktext}{HTML}{414141}
% \definecolor{text}{HTML}{333333}
% \definecolor{graytext}{HTML}{5D5D5D}
% \definecolor{lighttext}{HTML}{999999}

% Set false if you don't want to highlight section with awesome color
\setbool{acvSectionColorHighlight}{true}

% If you would like to change the social information separator from a pipe (|) to something else
\renewcommand{\acvHeaderSocialSep}{\quad\textbar\quad}

\def\endfirstpage{\newpage}

%-------------------------------------------------------------------------------
%	PERSONAL INFORMATION
%	Comment any of the lines below if they are not required
%-------------------------------------------------------------------------------
% Available options: circle|rectangle,edge/noedge,left/right

\name{Arran Hamlet}{}

\position{Postdoctural Researcher}
\address{Global Infectious Disease Analysis, School of Public Health,
Imperial College London}

\mobile{+44 7923340228}
\email{\href{mailto:arran.hamlet14@imperial.ac.uk}{\nolinkurl{arran.hamlet14@imperial.ac.uk}}}
\github{arranhamlet}
\linkedin{arranhamlet}

% \gitlab{gitlab-id}
% \stackoverflow{SO-id}{SO-name}
% \skype{skype-id}
% \reddit{reddit-id}


\usepackage{booktabs}

\providecommand{\tightlist}{%
	\setlength{\itemsep}{0pt}\setlength{\parskip}{0pt}}

%------------------------------------------------------------------------------



% Pandoc CSL macros
\newlength{\cslhangindent}
\setlength{\cslhangindent}{1.5em}
\newlength{\csllabelwidth}
\setlength{\csllabelwidth}{3em}
\newenvironment{CSLReferences}[3] % #1 hanging-ident, #2 entry spacing
 {% don't indent paragraphs
  \setlength{\parindent}{0pt}
  % turn on hanging indent if param 1 is 1
  \ifodd #1 \everypar{\setlength{\hangindent}{\cslhangindent}}\ignorespaces\fi
  % set entry spacing
  \ifnum #2 > 0
  \setlength{\parskip}{#2\baselineskip}
  \fi
 }%
 {}
\usepackage{calc}
\newcommand{\CSLBlock}[1]{#1\hfill\break}
\newcommand{\CSLLeftMargin}[1]{\parbox[t]{\csllabelwidth}{#1}}
\newcommand{\CSLRightInline}[1]{\parbox[t]{\linewidth - \csllabelwidth}{#1}}
\newcommand{\CSLIndent}[1]{\hspace{\cslhangindent}#1}

\begin{document}

% Print the header with above personal informations
% Give optional argument to change alignment(C: center, L: left, R: right)
\makecvheader

% Print the footer with 3 arguments(<left>, <center>, <right>)
% Leave any of these blank if they are not needed
% 2019-02-14 Chris Umphlett - add flexibility to the document name in footer, rather than have it be static Curriculum Vitae
\makecvfooter
  {January, 2021}
    {Arran Hamlet~~~·~~~Curriculum Vitae}
  {\thepage}


%-------------------------------------------------------------------------------
%	CV/RESUME CONTENT
%	Each section is imported separately, open each file in turn to modify content
%------------------------------------------------------------------------------



\hypertarget{education}{%
\section{Education}\label{education}}

\begin{cventries}
    \cventry{PhD in Infectious Disease Modelling}{Imperial College London}{London, United Kingdom}{2017-2020}{\begin{cvitems}
\item PhD thesis titled 'Yellow fever in South America: The role of environment and host on transmission dynamics'.
\item Project focused on understanding the epidemiology of yellow fever across South America, and in focus in Brazil, using climate, environment and host using a variety of statistical and mechanistic modelling techniques.
\item Awarded 3 Medical Research Council Exceptional Training Opportunity awards (total £) to present results as well as teach and run workshops in Brazil and Colombia to Ministries of Health and Universities.
\item Supervised by Dr Tini Garske and Professor Neil Ferguson.
\end{cvitems}}
    \cventry{MSc in Epidemiology}{Imperial College London}{London, United Kingdom}{2014-2015}{\begin{cvitems}
\item Studied a broad collection of topics before specialising in infectious disease epidemiology.
\item Awarded a distinction for my dissertation project titled 'The Seasonality of Yellow Fever in Africa.'
\end{cvitems}}
    \cventry{BSc in Biology with Psychology}{Queen Mary University of London}{London, United Kingdom}{2011-2014}{\begin{cvitems}
\item Graduated with Upper Second-class Honours (2.1)”.
\end{cvitems}}
\end{cventries}

\hypertarget{employment}{%
\section{Employment}\label{employment}}

\begin{cventries}
    \cventry{Postdoctural Researcher}{Imperial College London}{London, United Kingdom}{Jan 2020-}{\begin{cvitems}
\item Involved in a variety of projects assessing the public health impact of various control measures on the burden of malaria across Africa using mechanistic transmission models. 
\item Assessed the impact of disruption caused by the SARS-CoV-2 pandemic on malaria control across Africa. Results published Nature Medicine as joint first author.
\item Lead researcher on a piece of work commissioned by Abbott and the Presidents Malaria Initiative to assess the potential impact of Anopheles stephensi establishment on malaria transmission in the Horn of Africa.
\end{cvitems}}
    \cventry{PhD in Infectious Disease Modelling}{Imperial College London}{London, United Kingdom}{Jan 2017 - Jan 2020}{\begin{cvitems}
\item PhD thesis titled 'Yellow fever in South America: The role of environment and host on transmission dynamics'.
\end{cvitems}}
    \cventry{Research Assistant}{Imperial College London}{London, United Kingdom}{Oct 2015 - Dec 2016}{\begin{cvitems}
\item Outbreak analysis and response for the 2015-2016 outbreak in Angola and the Democratic Republic of the Congo working with the World Health Organization (WHO).
\item Responsible for estimating population-level vaccination coverage across Africa and the development of an open-source tool to explore this information. Currently utilised by researchers and the WHO.
\end{cvitems}}
\end{cventries}

\hypertarget{consultancy}{%
\section{Consultancy}\label{consultancy}}

\begin{cventries}
    \cventry{Epidemiologist: COVID-19}{World Health Organization}{Geneva, Switzerland}{Feb 2020 - Apr 2020}{\begin{cvitems}
\item Provided technical support for the WHO in the Health Emergency Information and Risk Assessment (HIM) pillar through GOARN deployment
\item Work involved exploring and quantifying mortality, transmission and country specific impacts through data analysis and visualisation in real time as the COVID-19 pandemic unfolded.
\end{cvitems}}
    \cventry{Epidemiologist: Yellow fever}{World Health Organization}{Geneva, Switzerland}{Jul 2016 - Sep 2016}{\begin{cvitems}
\item Commissioned to produce a report evaluating the risk of outbreaks of yellow fever across Africa as a result of ongoing transmission in Angola and the Democratic Republic of the Congo and the potential for introduction into Asia.
\end{cvitems}}
    \cventry{Epidemiologist}{Ozygen Systems}{London, United Kingdom}{Feb 2016 - Mar 2016}{\begin{cvitems}
\item Hired to produce a report on numerous pathogens involved in nosocomial infection and to evaluate the applicability of ozone decontamination technology in UK healthcare settings to limit their spread.
\end{cvitems}}
\end{cventries}

\hypertarget{teaching}{%
\section{Teaching}\label{teaching}}

\begin{cventries}
    \cventry{MSc dissertation supervisor}{Imperial College London}{London, United Kingdom}{May 2020 - Oct 2020}{\begin{cvitems}
\item Designed and supervised MSc Epidemiology projects looking at the effect of forest fragmentation on yellow fever in Southern Brazil, and exploring the differences in transmission dynamics between yellow fever, dengue and zika.
\end{cvitems}}
    \cventry{Graduate teaching assistant}{Imperial College London}{London, United Kingdom}{Jan 2017 - }{\begin{cvitems}
\item Teaching assistant and demonstrator for numerous modules on infectious disease modelling, statistical analysis and epidemiology.
\end{cvitems}}
    \cventry{Shortcourse demonstrator}{Imperial College London}{London, United Kingdom}{September 2017/September 2018/September 2019}{\begin{cvitems}
\item Demonstrator on the departments 'Mathematical modelling for the control of infectious diseases' short course, run since 1990 and designed to teach pubic health professionals about infectious disease modelling.
\end{cvitems}}
    \cventry{Shortcourse demonstrator}{Imperial College London}{Bogota, Colombia}{June 2020}{\begin{cvitems}
\item Demonstrator and lecturer on a course coordinated between Imperial College London, Instituto Nacional De Salud and Pontificia Universidad Javeriana Bogota which aimed to give an introduction to infectious disease modelling.
\end{cvitems}}
    \cventry{Design and implementation of an online platform for teaching infectious disease modelling}{Imperial College London}{London, United Kingdom}{Jan 2019 - Sept 2019}{\begin{cvitems}
\item Responsible for liasing between programming team and course organisers to design and translate existing practicals from Berkely Madonna to an online web interface running the Odin language.
\item Highly successful implementation with the platform now being used for both future shortcourses and the MSc Epidemiology at Imperial College London.
\end{cvitems}}
\end{cventries}

\hypertarget{publications}{%
\section{Publications}\label{publications}}

\hypertarget{bibliography}{}
\leavevmode\hypertarget{ref-R-vitae}{}%
1. O’Hara-Wild, M., \& Hyndman, R. (2020). \emph{Vitae: Curriculum
vitae for r markdown}. \url{https://CRAN.R-project.org/package=vitae}

\leavevmode\hypertarget{ref-R-tibble}{}%
2. Müller, K., \& Wickham, H. (2020). \emph{Tibble: Simple data
frames}. \url{https://CRAN.R-project.org/package=tibble}

\end{document}
