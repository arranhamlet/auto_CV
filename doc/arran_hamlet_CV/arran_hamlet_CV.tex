%!TEX TS-program = xelatex
%!TEX encoding = UTF-8 Unicode
% Awesome CV LaTeX Template for CV/Resume
%
% This template has been downloaded from:
% https://github.com/posquit0/Awesome-CV
%
% Author:
% Claud D. Park <posquit0.bj@gmail.com>
% http://www.posquit0.com
%
%
% Adapted to be an Rmarkdown template by Mitchell O'Hara-Wild
% 23 November 2018
%
% Template license:
% CC BY-SA 4.0 (https://creativecommons.org/licenses/by-sa/4.0/)
%
%-------------------------------------------------------------------------------
% CONFIGURATIONS
%-------------------------------------------------------------------------------
% A4 paper size by default, use 'letterpaper' for US letter
\documentclass[11pt, a4paper]{awesome-cv}

% Configure page margins with geometry
\geometry{left=1.4cm, top=.8cm, right=1.4cm, bottom=1.8cm, footskip=.5cm}

% Specify the location of the included fonts
\fontdir[fonts/]

% Color for highlights
% Awesome Colors: awesome-emerald, awesome-skyblue, awesome-red, awesome-pink, awesome-orange
%                 awesome-nephritis, awesome-concrete, awesome-darknight

\colorlet{awesome}{awesome-red}

% Colors for text
% Uncomment if you would like to specify your own color
% \definecolor{darktext}{HTML}{414141}
% \definecolor{text}{HTML}{333333}
% \definecolor{graytext}{HTML}{5D5D5D}
% \definecolor{lighttext}{HTML}{999999}

% Set false if you don't want to highlight section with awesome color
\setbool{acvSectionColorHighlight}{true}

% If you would like to change the social information separator from a pipe (|) to something else
\renewcommand{\acvHeaderSocialSep}{\quad\textbar\quad}

\def\endfirstpage{\newpage}

%-------------------------------------------------------------------------------
%	PERSONAL INFORMATION
%	Comment any of the lines below if they are not required
%-------------------------------------------------------------------------------
% Available options: circle|rectangle,edge/noedge,left/right

\name{Dr Arran Hamlet}{}

\position{Postdoctoral Researcher}
\address{Global Infectious Disease Analysis, School of Public Health,
Imperial College London}

\email{\href{mailto:arran.hamlet14@imperial.ac.uk}{\nolinkurl{arran.hamlet14@imperial.ac.uk}}}
\github{arranhamlet}
\linkedin{arranhamlet}

% \gitlab{gitlab-id}
% \stackoverflow{SO-id}{SO-name}
% \skype{skype-id}
% \reddit{reddit-id}

\quote{Infectious disease epidemiologist and mathematical modeller with
extensive experience in data analysis and statistical/mechanistic
modelling to inform outbreak response and public health policy. Strong
interest in work within LMIC settings and the programming of interactive
tools for data analysis and decision making.}

\usepackage{booktabs}

\providecommand{\tightlist}{%
	\setlength{\itemsep}{0pt}\setlength{\parskip}{0pt}}

%------------------------------------------------------------------------------



% Pandoc CSL macros
\newlength{\cslhangindent}
\setlength{\cslhangindent}{1.5em}
\newlength{\csllabelwidth}
\setlength{\csllabelwidth}{3em}
\newenvironment{CSLReferences}[3] % #1 hanging-ident, #2 entry spacing
 {% don't indent paragraphs
  \setlength{\parindent}{0pt}
  % turn on hanging indent if param 1 is 1
  \ifodd #1 \everypar{\setlength{\hangindent}{\cslhangindent}}\ignorespaces\fi
  % set entry spacing
  \ifnum #2 > 0
  \setlength{\parskip}{#2\baselineskip}
  \fi
 }%
 {}
\usepackage{calc}
\newcommand{\CSLBlock}[1]{#1\hfill\break}
\newcommand{\CSLLeftMargin}[1]{\parbox[t]{\csllabelwidth}{#1}}
\newcommand{\CSLRightInline}[1]{\parbox[t]{\linewidth - \csllabelwidth}{#1}}
\newcommand{\CSLIndent}[1]{\hspace{\cslhangindent}#1}

\begin{document}

% Print the header with above personal informations
% Give optional argument to change alignment(C: center, L: left, R: right)
\makecvheader

% Print the footer with 3 arguments(<left>, <center>, <right>)
% Leave any of these blank if they are not needed
% 2019-02-14 Chris Umphlett - add flexibility to the document name in footer, rather than have it be static Curriculum Vitae
\makecvfooter
  {February, 2021}
    {Dr Arran Hamlet~~~·~~~Curriculum Vitae}
  {\thepage}


%-------------------------------------------------------------------------------
%	CV/RESUME CONTENT
%	Each section is imported separately, open each file in turn to modify content
%------------------------------------------------------------------------------



\hypertarget{qualifications}{%
\section{Qualifications}\label{qualifications}}

\begin{cventries}
    \cventry{Imperial College London}{PhD titled 'Yellow fever in South America: The role of environment and host on transmission dynamics'}{2017-2020}{London, United Kingdom}{\begin{cvitems}
\item Project focused on understanding the epidemiology of yellow fever across South America, with a focus on Brazil, examining the roles of climate, environment and host, using a variety of statistical and mechanistic modelling techniques.
\item Awarded 3 Medical Research Council (MRC) Exceptional Training Opportunity awards (total £4,589) and a MRC pump priming award (£23,000) to present results as well as teach, design and run workshops in Brazil and Colombia for Ministries of Health and Universities.
\item Supervised by Dr Tini Garske and Professor Neil Ferguson, fully funded by the UK Medical Research Council.
\end{cvitems}}
    \cventry{Imperial College London}{MSc in Epidemiology}{2014-2015}{London, United Kingdom}{\begin{cvitems}
\item Studied a broad collection of topics before specialising in infectious disease epidemiology.
\item Awarded a distinction for my dissertation project titled 'The Seasonality of Yellow Fever in Africa.'
\end{cvitems}}
    \cventry{Queen Mary University of London}{BSc in Biology with Psychology}{2011-2014}{London, United Kingdom}{\begin{cvitems}
\item Graduated with Upper Second-class Honours (2.1).
\end{cvitems}}
\end{cventries}

\hypertarget{employment}{%
\section{Employment}\label{employment}}

\begin{cventries}
    \cventry{Imperial College London}{Postdoctoral Researcher (Malaria)}{Jan 2020-}{London, United Kingdom}{\begin{cvitems}
\item Involved in a variety of projects assessing the public health impact of various control measures on the burden of malaria across Africa using mechanistic transmission models. 
\item Assessed the impact of disruption caused by the SARS-CoV-2 pandemic on malaria control across Africa. Results published Nature Medicine as joint first author.
\item Lead researcher on a piece of work commissioned by Abbott and the Presidents Malaria Initiative to assess the potential impact of Anopheles stephensi establishment on malaria transmission in the Horn of Africa.
\end{cvitems}}
    \cventry{Imperial College London}{Postdoctoral Researcher (Nigeria COVID-19 response)}{Apr 2020-}{London, United Kingdom}{\begin{cvitems}
\item Lead researcher for Imperial College London's data analytics and modelling support for Nigeria.
\item Conducted analysis to answer specific questions in order to provide evidence for decisions to be made by the Nigerian Presidential Task Force.
\item Produced a multitidue of reports as well as regular state-specicic analysis that fed into NCDC, US CDC and UK Department for International Development decision making.
\item A number of position papers can be found at https://statehouse.gov.ng/covid19/2020/09/18/evidence-based-guidance-on-measures-to-curb-the-spread-of-covid-19/.
\end{cvitems}}
    \cventry{Imperial College London}{Postdoctoral Researcher (COVID-19 response)}{Feb 2020-}{London, United Kingdom}{\begin{cvitems}
\item Provided technical support and input for numerous reports and projects, with a focus on work in Low-to-Middle-Income Countries (LMIC) and on quantifying the underascertainment of mortality.
\item Seconded through the Global Outbreak Alert and Response Network to provide technical support for the WHO in Geneva Feb - Apr 2020.
\end{cvitems}}
    \cventry{Imperial College London}{PhD in Infectious Disease Modelling}{Jan 2017 - Jan 2020}{London, United Kingdom}{\begin{cvitems}
\item PhD thesis titled 'Yellow fever in South America: The role of environment and host on transmission dynamics'.
\end{cvitems}}
    \cventry{Imperial College London}{Research Assistant}{Oct 2015 - Dec 2016}{London, United Kingdom}{\begin{cvitems}
\item Outbreak analysis and response for the 2015-2016 outbreak in Angola and the Democratic Republic of the Congo working with the World Health Organization (WHO).
\item Responsible for estimating population-level vaccination coverage across Africa and the development of an open-source tool to explore this information. Currently utilised by researchers and the WHO.
\end{cvitems}}
\end{cventries}

\hypertarget{consultancy}{%
\section{Consultancy}\label{consultancy}}

\begin{cventries}
    \cventry{World Health Organization}{Epidemiologist: COVID-19}{Feb 2020 - Apr 2020}{Geneva, Switzerland}{\begin{cvitems}
\item Provided technical support for the WHO in the Health Emergency Information and Risk Assessment (HIM) pillar through GOARN deployment
\item Work involved exploring and quantifying mortality, transmission and country specific impacts through data analysis and visualisation in real time as the COVID-19 pandemic unfolded. Aspects of data visualisation acknowledged in https://worldhealthorg.shinyapps.io/covid/.
\item Continued to provide adhoc support till Dec 2020.
\end{cvitems}}
    \cventry{World Health Organization}{Epidemiologist: Yellow fever}{Jul 2016 - Sep 2016}{Geneva, Switzerland}{\begin{cvitems}
\item Commissioned to produce a report evaluating the risk of outbreaks of yellow fever across Africa as a result of ongoing transmission in Angola and the Democratic Republic of the Congo and the potential for introduction into Asia.
\end{cvitems}}
    \cventry{Ozygen Systems}{Epidemiologist}{Feb 2016 - Mar 2016}{London, United Kingdom}{\begin{cvitems}
\item Hired to produce a report on numerous pathogens involved in nosocomial infection and to evaluate the applicability of ozone decontamination technology in UK healthcare settings to limit their spread.
\end{cvitems}}
\end{cventries}

\hypertarget{teaching}{%
\section{Teaching}\label{teaching}}

\begin{cventries}
    \cventry{University of São Paulo}{PhD Assessor}{Oct 2020-}{São Paulo, Brazil}{\begin{cvitems}
\item Examining progress and assisting with the research of a PhD student's project titled 'Spatio-temporal dynamics of yellow fever in Brazil'.
\end{cvitems}}
    \cventry{Imperial College London}{MSc Dissertation Supervisor}{May 2020 - Oct 2020}{London, United Kingdom}{\begin{cvitems}
\item Designed and supervised MSc Epidemiology projects looking at the effect of forest fragmentation on yellow fever in Southern Brazil, and exploring the differences in transmission dynamics between yellow fever, dengue and zika.
\end{cvitems}}
    \cventry{Imperial College London}{Graduate Teaching Assistant}{Jan 2017 - }{London, United Kingdom}{\begin{cvitems}
\item Teaching assistant and demonstrator for numerous modules on infectious disease modelling, statistical analysis and epidemiology.
\end{cvitems}}
    \cventry{Imperial College London}{Shortcourse Demonstrator}{Sep 2017/Sep 2018/Sep 2019}{London, United Kingdom}{\begin{cvitems}
\item Demonstrator on the departments 'Mathematical modelling for the control of infectious diseases' short course, run since 1990 and designed to teach pubic health professionals about infectious disease modelling.
\end{cvitems}}
    \cventry{Imperial College London}{Shortcourse Demonstrator}{Jun 2020}{Bogota, Colombia}{\begin{cvitems}
\item Demonstrator and lecturer on a course coordinated between Imperial College London, Instituto Nacional De Salud and Pontificia Universidad Javeriana Bogota which aimed to give an introduction to infectious disease modelling.
\end{cvitems}}
    \cventry{Imperial College London}{Design and implementation of an online platform for teaching infectious disease modelling}{Jan 2019 - Sept 2019}{London, United Kingdom}{\begin{cvitems}
\item Responsible for liasing between programming team and course organisers to design and translate existing practicals from Berkely Madonna to an online web interface running the Odin language.
\item Highly successful implementation with the platform now being used for both future shortcourses and the MSc Epidemiology at Imperial College London.
\end{cvitems}}
\end{cventries}
\newpage

\hypertarget{funding-awards}{%
\section{Funding awards}\label{funding-awards}}

\begin{cventries}
    \cventry{Imperial College London}{MRC Pump Priming (£23,000)}{Nov 2019}{London, United Kingdom}{\begin{cvitems}
\item Jointly awarded £23,000 with co-PI Natsuko Imai to run a week long training workshop in Rio de Janeiro focusing on the use of mathematical models in outbreak response and policy in July 2020.
\item Course was to be run collaboratively with the Brazilian Ministry of Health and Fundação Oswaldo Cruz (Fiocruz).
\item Postponed due to the COVID-19 pandemic.
\end{cvitems}}
    \cventry{Imperial College London}{MRC Exceptional Training Opportunity}{Oct 2017/Aug 2018/Jun 2019}{London, United Kingdom}{\begin{cvitems}
\item Oct 2017: Awarded £650 to travel to the WHO in Geneva, Switzerland to present my work on yellow fever and discuss with the yellow fever team how my PhD can provide support for their activities.
\item Aug 2018: Awarded £2220 to travel to Rio de Janiro and Brasilia, Brazil, and present my results on modelling yellow fever in South America at a meeting co-hosted by the Brazilian Ministry of Health and the Pan American Health Organization, as well as to set up a research collaboration with Fiocruz.
\item Jun 2019: Awarded £1719 to travel to Bogota, Colombia to lecture and demonstrate on a course coordinated between Imperial College London, Instituto Nacional De Salud and Pontificia Universidad Javeriana Bogota which aimed to give an introduction to infectious disease modelling.
\end{cvitems}}
\end{cventries}

\hypertarget{presentations}{%
\section{Presentations}\label{presentations}}

\begin{cventries}
    \cventry{London Malaria Network}{The potential public health consequences of COVID-19 on malaria in Africa}{Oct 2020}{London, United Kingdom}{}\vspace{-4.0mm}
    \cventry{American Society of Tropical Medicine and Hygiene}{Seasonality of agricultural exposure more important than seasonality of climate for predicting yellow fever transmission in Brazil}{Nov 2019}{National Harbor, United States of America}{}\vspace{-4.0mm}
    \cventry{Outbreak Analysis and Modelling for Public Health}{Statistical and mathematical modelling of yellow fever in South America}{Jun 2019}{Bogota, Colombia}{}\vspace{-4.0mm}
    \cventry{International Conference on One Medicine One Science}{Land-use,  vegetation  and  habitat fragmentation  as  drivers  of  yellow  fever  transmission  in  South America}{Feb 2019}{Chang Mai, Thailand}{}\vspace{-4.0mm}
    \cventry{Yellow fever forecasting: Embedding   modelling in lessons learnt exercises}{Yellow fever in Brazil - Modelling as a tool to inform outbreak response and public health policy}{Nov 2018}{Brasilia, Brazil}{}\vspace{-4.0mm}
\end{cventries}

\hypertarget{publications}{%
\section{Publications}\label{publications}}

\hypertarget{first-author-publications}{%
\subsection{First author publications}\label{first-author-publications}}

\begin{cventries}
    \cventry{Seasonal and inter-annual drivers of yellow fever transmission in South America}{A Hamlet, KAM Gaythorpe, T Garske, NM Ferguson}{PLoS neglected tropical diseases}{2021}{}\vspace{-4.0mm}
    \cventry{POLICI: A web application for visualising and extracting yellow fever vaccination coverage in Africa}{A Hamlet, K Jean, S Yactayo, J Benzler, L Cibrelus, N Ferguson, T Garske}{Vaccine}{2019}{}\vspace{-4.0mm}
    \cventry{The seasonal influence of climate and environment on yellow fever transmission across Africa}{A Hamlet, K Jean, W Perea, S Yactayo, J Biey, M Van Kerkhove, ...}{PLoS neglected tropical diseases}{2018}{}\vspace{-4.0mm}
\end{cventries}

\hypertarget{selected-additional-publications}{%
\subsection{Selected additional
publications}\label{selected-additional-publications}}

\begin{cventries}
    \cventry{The impact of COVID-19 and strategies for mitigation and suppression in low-and middle-income countries}{PGT Walker, C Whittaker, OJ Watson, M Baguelin, P Winskill, A Hamlet, ...}{Science}{2020}{}\vspace{-4.0mm}
    \cventry{The potential public health consequences of COVID-19 on malaria in Africa}{E Sherrard-Smith, AB Hogan, A Hamlet, OJ Watson, C Whittaker, ...}{Nature medicine}{2020}{}\vspace{-4.0mm}
    \cventry{The effect of climate change on yellow fever disease burden in Africa}{KAM Gaythorpe, A Hamlet, L Cibrelus, T Garske, NM Ferguson}{Elife}{2020}{}\vspace{-4.0mm}
    \cventry{Eliminating yellow fever epidemics in Africa: Vaccine demand forecast and impact modelling}{K Jean, A Hamlet, J Benzler, L Cibrelus, KAM Gaythorpe, A Sall, ...}{PLoS neglected tropical diseases}{2020}{}\vspace{-4.0mm}
    \cventry{International risk of yellow fever spread from the ongoing outbreak in Brazil, December 2016 to May 2017}{I Dorigatti, A Hamlet, R Aguas, L Cattarino, A Cori, CA Donnelly, T Garske, ...}{Eurosurveillance}{2017}{}\vspace{-4.0mm}
\end{cventries}

\hypertarget{selected-reports}{%
\subsection{Selected reports}\label{selected-reports}}

\begin{cventries}
    \cventry{Report 9: Impact of non-pharmaceutical interventions (NPIs) to reduce COVID19 mortality and healthcare demand}{N Ferguson, D Laydon, G Nedjati Gilani, N Imai, K Ainslie, M Baguelin, ...}{}{2020}{}\vspace{-4.0mm}
    \cventry{Report 12: The global impact of COVID-19 and strategies for mitigation and suppression}{P Walker, C Whittaker, O Watson, M Baguelin, K Ainslie, S Bhatia, S Bhatt, ...}{}{2020}{}\vspace{-4.0mm}
    \cventry{Report 4: severity of 2019-novel coronavirus (nCoV)}{I Dorigatti, L Okell, A Cori, N Imai, M Baguelin, S Bhatia, A Boonyasiri, ...}{}{2020}{}\vspace{-4.0mm}
    \cventry{Report 16: Role of testing in COVID-19 control}{N Grassly, M Pons Salort, E Parker, P White, K Ainslie, M Baguelin, ...}{}{2020}{}\vspace{-4.0mm}
    \cventry{Report 8: Symptom progression of COVID-19}{K Gaythorpe, N Imai, G Cuomo-Dannenburg, M Baguelin, S Bhatia, ...}{}{2020}{}\vspace{-4.0mm}
\end{cventries}

\end{document}
